\section{Introduction}

Titan is the only moon of the solar system with a thick hazy atmosphere which represents about 20\% of its apparent
diameter. Its atmosphere is mainly composed of nitrogen and methane. The photo-dissociation of these molecules by the
UV light in the upper part of the atmosphere leads to the production a large number other hydrocarbons and nitriles as
trace species and to photochemical haze of aerosols. This haze is global and completely covers Titan. It controls
the thermal balance through its visible and thermal infrared properties. It also veils the lower atmosphere and the
surface that can be perceived in few methane windows in near infrared.

Although the haze was detected in the middle 70's, it was first imaged by Pioneer 11 and the two Voyagers
\citep{Smith1981, Smith1982, Sromovsky1981}. Titan's haze layer had several remarkable structures as a northern (winter)
polarhood, an interhemispheric asymmetry and a thin global detached haze layer above the main global haze. It was
readily though that the detached haze layer had a dynamical origin \citep{Smith1981}. Photometric analysis allowed
to give the extinction properties of both hazes and to evaluated the effective radius of the aerosols in the detached
haze ($\simeq 0.3 \mu$m) and in the main haze ($\simeq 0.4 \mu$m) \citep{Rages1983, Rages1983a}. Analysis show
that the detached haze layer appears due to a strong depletion in aerosols extinction around 300 km, yielding a distinct
layer above the main haze with a maximum extinction located around 350 km \citep{Rages1983}. Its horizontal extent is
very stable in pressure and it is reported at all the southern latitudes up to \ang{45}N where it connects to the northern
polar hood. The detached haze layer could be seen again twenty years after the Voyager flybys \citep{Porco2005}. The
main change was in its altitude at 515 km, that is 165 km higher than in 1981. Again, it was a fairly homogeneous global
shell above the main haze at a constant altitude. The DHL was merging with the northern polarhood. Notably, while Voyager
observations were performed after the northern spring equinox, Cassini early observations occurred during the northern
winter, that is half a season before the next northern spring equinox.

The first attempt to explain the observation of the DHL is proposed by \cite{Toon1992}. They used a 1D microphysical
model where a {\em ad hoc} vertical wind is added to maintain aloft the particles at a constant altitude above the main
haze. Alternative scenarii were proposed to explain the DHL from pure microphysical processes. \cite{Chassefiere1995}
investigated the case of two different aerosol production layers. They proposed that the uppermost layer (500-1000 km)
produces fluffy aggregates that could be swept horizontally by winds, generating a detached haze layer. They also
propose an alternative scenario where aerosols settle downward and interact with macromolecules from the main haze,
produced by the lower production zone (around 350-400 km). In the latter case, the interaction would produce by some
way an optical gap. However, they clearly favorize the scenario involving winds which would match all the constraints
known at that time. In the same vein, \cite{Lavvas2009} proposed a scenario based on a pure microphysical process. Aerosols
are produced at high altitude (as in \cite{Chassefiere1995} hypothesis) growing as spheres down to levels around 500 km.
But here, the detached haze is produced by a sudden change in the fractal dimension of the aerosols. This produces an
equivalent sudden change in the microphysical laws, and an artificial optical gap. However, this change in fractal
dimension is not explained and they find an aerosol radius of 40 nm in the detached haze, which does not match the data.

Later, with a 2D-General Climate Model (GCM) accounting for the transport of haze by dynamics and the radiative
feedback it was possible to reproduce and explain the mechanism that produces the DHL \citep{Rannou2002}. It was also
demonstrated that this feedback strongly enhances the wind speed due to the thick polar haze cap at winter pole built
by the circulation. In return, this cap enhances the cooling to space during the polar night \citep{Rannou2004} an
reinforce the circulation. Due to Titan's obliquity (\ang{27}) and the slow diurnal rotation rate, seasons are well
marked and circulation cells span over all the planet due to geostrophy. This situation leads to the formation of a
broad ascending circulation in the summer hemisphere able to lift aerosols up to high altitudes where they remain
suspended and are transported through  mid-latitudes to the winter polar region where they are transported by
subsidence. In this scenario, the location of the DHL corresponds to the area where the settling speed is compensated
by upward wind and evolves with the changes of illumination along the seasons. More sophisticated 3D-GCM improved the
understanding of the haze cycle, including the formation of the detached haze, and basically confirmed this picture
\citep{Lebonnois2012,Larson2015}. This formation mechanism implies that the DHL is a blending of aerosols newly
produced and falling from above and older and larger aerosols produced in the stratosphere and lifted by circulation.
Although the GCM's results differ in some aspects with observation, they are able to capture the main picture behind
the existence of the DHL.

Photometric studies performed with Cassini observations taken before 2009 equinox combined complementary observations.
In one hand, intensities scattered at the limb in UV (340 nm) at different phase angles measured by ISS were used. On
the other hand, a single value of the tangential opacity in VUV (187 nm) retrieved from UVIS observations in the
occultation mode \citep{Koskinen2011} was added to the study. The result shows that large aerosols with an effective bulk
radius $\simeq$ 0.2 $\mu$m produces all the scattering while small nanometric aerosols are needed to explain most of the
extinction \citep{Cours2011, Seignovert2017}. This is quite consistent with a DHL made of two different populations of
aerosols.

\cite{West2011} reported a rapid collapse of the detached haze starting just after the equinox. The altitude of
the DHL descended by about 80 km in 200 terrestrial days and by 30 km more in about 300 terrestrial days. A simple
extrapolation of the altitude of the DHL with time indicated that it would be at the same altitude as observed by Voyager
exactly one Titan year after the Voyager epoch. \cite{West2011} concluded that such a result was coherent with a seasonal
cycle of the detached haze.  They compared their results with a 2D-GCM and made a prediction about the reappearance of the
DHL several years later (2013-2016) at its initial altitude (around 500 km). \cite{Lebonnois2012} and \cite{Larson2015} made
similar predictions but with an reappearance of the DHL a bit later, around the next northern summer solstice ($L_S=\ang{95}$
and $\ang{70}-\ang{80}$ respectively). In reality, \cite{West2018} found that the DHL reappeared in early 2016
($L_S=\ang{73}-\ang{76}$) at 480 km, several months before the solstice. They followed the cycle of the DHL at the equator
and retrieved the haze extinction
profile in the CL1-UV3 filter combination. Its reappearance was much more complex than predicted. This first detached haze
dropped in altitude down to 470 km within a terrestrial year and vanished while a new DHL emerged again around 500 km. This
new layer appeared quite stable until the end of the Cassini mission (september 2017 - $L_S=\ang{91}$). Unfortunately, no other
mean exists to further probe the DHL and nothing is known about the fate of the detached haze after this date.

In the present work, we perform a systematic latitude-altitude mapping of the detached haze layer between 350 to 600 km.
We used observations made by the narrow angle camera (NAC) of ISS in the filter combination CL1-UV3 during all the
mission. That covers the period between July 2004 (half a season after the northern winter solstice) and the end of
the mission in september 2017 (after the summer solstice). We used exactly the same model as \cite{West2018}, that is
a ray tracing model in spherical shell for the single scattering albedo and a correction for the multiples scattering.

The outline of the article is as follow. In the next section (\textbf{section 2}), we first give a global presentation of
available data and the criterion used to select data used in this paper.  Then we describe the main principle of the
retrieval model that is used for the analysis and the the retrieval method. The \textbf{section 3} is the core of the
article. We present the results of the photometric analysis as latitude - altitude panels showing the spatial distribution
of the detached haze and the upper part of the main haze. The seasonal cycle of the DHL is split in four specific
periods between 2004 and 2017. This section has then 4 subsections for each periods where we explain in detail the main
characteristics of the haze and its evolution. The \textbf{fourth section} is dedicated to the study of specific sets of
observations that allow to probe short time, short term or diurnal variations. We first describe how the data were selected
and then what they reveal about Titan atmosphere. In the \textbf{section 5}, we make comparisons between our results and
results obtained at the same location and the same time with UVIS \textcolor{blue}{Remove ? : and CIRS, both} onboard Cassini. We
also make comparisons between our results and prediction made by two Titan 3D-GCM about the detached haze layer and its evolution.
The conclusion and the perspective of this work are given in \textbf{section 6}.
