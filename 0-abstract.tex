\begin{abstract}

\change{This study presents a 13 years survey of haze UV extinction profiles, monitoring the temporal evolution of the detached haze layer (DHL) in Titan’s upper atmosphere (350-600 km). As reported by West et al. 2011 (GRL vol.38, L06204) at the equator, we show that the DHL is present at all latitudes below \ang{55}N during the northern winter (2004-2009). Then, it globally sunk and disappeared in 2012. No permanent DHL was observed between 2012 and 2015. It’s only in late-2015, that a new structure emerged from the Northern hemisphere and propagated to the equator. This new DHL is not as pronounced as in 2004 and is much more complex than the one observed earlier.
In one specific sequence, in 2005, we were able to investigate the short time scale variability of the DHL and no major changes was observed.
When both side of the limb were visible (dawn/dusk), we notice that the extinction of the DHL is slightly higher on the dawn side.
Additionally, during a polar flyby in 2009, we observed the longitudinal variability of the DHL and spotted some local inhomogeneities.
Finally, comparisons with UVIS stellar occultations and General Climate Models (GCMs) are both consistent with our findings.
However, we noticed that the timing of the DHL main pattern predicted by the GMCs can be off by up to \ang{30} in solar longitude.
All these observations bring new perspectives on the seasonal cycle of Titan's upper atmosphere, the evolution of the DHL and its interaction with the dynamics.}

\end{abstract}

\keywords{Cassini --- Titan --- Haze --- Seasonal cycle}
