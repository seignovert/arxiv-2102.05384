The 13 years of the Cassini mission provide a unique dataset monitoring
the evolution of Titan’s atmosphere during almost half a Titan's year,
between the middle of the northern winter up to the northern summer solstice.
In this work we present a survey of the haze vertical extinction profiles retrieved
between 300 to 700 km with an atmospheric limb multiple scattering model.
We analyze 139 ISS NAC CL1-UV3 images taken between 2004 to 2017 covering the
entire Cassini mission.
We mainly focus our attention on the temporal evolution of the haze content
in the upper atmosphere to track the evolution of the detached haze layer.
As previously reported by \cite{West2011}, during the northern winter the detached haze layer is present
at all latitudes below \ang{60}N and globally collapse on the main haze
before the vernal equinox in 2009.
No permanent detached haze layer was observed between 2012 and 2015 but
we show that at the end of 2015, a new structure emerged from
the summer hemisphere (North) and propagated to the equator. This new layer
was not as pronounced and was much more complex than the one observed at the
beginning of the mission and was likely to be the initiation of a new detached haze layer.
We also investigate the short time scale variability of the detached haze
layer and no major changes is observed. Finally, we report some cases
where the viewing geometry allows us to probe the longitudinal variability
of the haze, highlighting some local inhomogeneities.
We notice that the extinction in the detached haze layer is slightly higher in the dawn
side compare to the dusk side.
All these observations bring new perspectives
on the seasonal cycle of Titan's upper atmosphere, the evolution of the detached
haze layer and its interaction with the dynamics.
The main seasonal pattern predicted by the global circulation models is consistent
with our findings, but the details and the timing of the global collapse
is not yet explained. Finally, comparisons with the UVIS instrument shows
that the haze extinction profiles retrieved during two stellar occultations
in 2009 are consistent with our results.