\subsection{Comparison with UVIS occultations}

\cite{Koskinen2011} performed a complete analysis of the mesosphere and thermosphere of Titan using UVIS stellar
occultations. The sensitivity of UVIS during a stellar occultation is much better than what ISS can achieve. However,
while ISS probes the light scattered by the detached haze layer, UVIS probes the light transmitted through a tangential
path at the limb. In both cases, an extinction profile can be retrieved. ISS can only retrieve the extinction of the
particles which scatters light, and under assumptions concerning the phase function and the single scattering albedo.
On the other hand, UVIS is able to retrieve the total extinction from transmission with no assumption about the haze
particles. In theory, this difference is valuable because it would give a unique information about the change in aerosol
size with altitude. In practice, ISS sensitivity is not sufficient to reach the requested sensitivity to probe above
the peak of the detached haze by more than a scale height.

Two of the UVIS occultation profiles, in 2008 and 2009 (T41 and T53 flybys), can be directly compared with ISS
observations at the same location and at the same period (Fig.~\ref{fig:uvis_iss}). The UVIS profiles are scaled to
offset the spectral dependence between ISS and UVIS effective wavelength (338 nm and 1850-1900~\AA~respectively).
This offset is due to the spectral dependence of the extinction cross-sections and to the intrinsic
differences arising from comparing the extinction retrieved from scattering properties or from occultation
\citep[see.][]{Cours2011}.

\begin{figure}[!ht]
    \centering
    \includegraphics[width=.8\textwidth]{Fig/UVIS_ISS.png}
    \caption[UVIS comparison]{Comparisons between ISS and UVIS extinction profiles before the equinox.
            UVIS profiles are retrieved by \cite{Koskinen2011} during the T41 (2008/02/23) and T53 (2009/04/19) flybys.
            ISS profiles are retrieved for the images \textbf{N1585329510\_1} (2008/03/27) and
            \textbf{N1618568958\_1} (2009/04/16).
            The UVIS profile are scaled by a factor 0.15 to compensate the spectral dependence of the extinction
            cross section and overlap ISS retrievals.}
    \label{fig:uvis_iss}
\end{figure}

The comparisons between the two extinction profiles show a very good match between the two extinction profiles
between 450 and 550 km. Considering that UVIS and ISS profiles are not taken simultaneously and
they don't probe the same longitude, the results of the previous section demonstrate that these differences are
consistent with the natural variabilities observed in the detached haze layer.
This comparison is then a good validation of our results concerning the extinction profiles of the detached haze.

Above 575 km, UVIS extinction profiles show the presence of a secondary layer at 610 km which is not detected
by ISS. As shown by \cite{Cours2011}, ISS is only sensitive to the larger aerosols, those that scatter light, whereas
UVIS is able to probe the extinction of all the particles, and especially the smaller ones which do not scatter.
In theory, the difference between UVIS and ISS above 575 km may reveal a sharp change in aerosol size distribution.
But, this layer is located at altitudes where the signal to noise is low and structure may be erased during the
deconvolution process. Therefore, it is not possible to draw a safe conclusion from the profiles above 575 km.
