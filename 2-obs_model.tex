\section{Observations and models}
\label{seq:observations}

\subsection{Selection of observations}
We conduct our survey on 138 images\footnote{See Data Availability section to get the list and the results for all the ISS images analyzed.}
taken by the Cassini Image Science Sub-System Narrow Angle Camera (ISS-NAC) with the
clear and ultra-violet filter combination CL1-UV3. At this wavelength (338 nm), ISS is sensitive to haze in the
upper part of Titan's atmosphere (300-600 km altitude). We choose the best sample among the 317 images available on the NASA Planetary Data System (PDS)
to get the highest temporal and phase coverage. For the main study we kept only images taken with at least a one-day gap.
We also kept specific sets of observations made a few hours apart to study short-term variations.

On average, the selected images are separated by 39 Earth days, \ie 2.5 Titan days (Fig.~\ref{fig:img_sampling}).
Although our sampling is not evenly distributed, due to orbital constraints and mission schedule, at least 90\% of the selected
images are separated by less than 120 Earth days, \ie 7.5 Titan days. Two main gaps of data can be observed.
The first is between 28 March 2008 and 25 January 2009 (302 Earth days/19 Titan days) and 26 November 2010 and 9 September 2011 (286 Earth days/18 Titan days).

\begin{figure*}[ht!]
\plotone{Fig/IMG_interval}
\caption{Timeline of the ISS/NAC CL1-UV3 images analyzed (vertical black ticks). The number of observations per 2 Earth months period (\ie 4 Titan days / $\Delta Ls \approx \ang{2}$) is reported with a green gradient scale. The gray area correspond to the gaps with no available observations.}
\label{fig:img_sampling}
\end{figure*}

The selected images are calibrated using the CISSCAL routines (v3.8) provided on the Planetary Data System. To improve
the signal to noise ratio on the limb profile, we deconvolved the images with a Poisson Maximum A Posteriori
(PMAP) method using the point spread function (PSF) measured in-flight \citep{West2010, Knowles2020}.
This deconvolution method is known to be efficient to restore fine strutures in astronomical images \citep{Hunt1996}.
In our case, the main source of light is always Titan and is resolved inside the field of view of the camera (Fig.~\ref{fig:model_uncertainties}a). Therefore, we don't expect to see a significant contribution of the stray light in this configuration \citep{West2010}.
The image pointing is initialized with the SPICE kernels \citep{Acton1996, Annex2020}. Since the surface of Titan is not visible in the UV, we improved
the location of Titan's center by fitting the limb intensity. Then, we calculated the planetocentric coordinates of each
pixel and their tangent point altitude with respect to a mean spherical body with a radius of 2575 km.

Intensity profiles are extracted every \ang{5} on both sides of the limb. Depending on the
latitude of the Sub-Cassini point on the ground, the sampling in latitude is not evenly distributed for each image.
Therefore, the polar latitudes are usually less covered than the equator.
Also, the solar illumination changes drastically during the
season between the northern mid-winter to summer, which restricts our ability to see both poles at the same time.
On average, the image pixel scale is about $10 \pm 5$~km.


\subsection{Model of scattering at the limb}

To retrieve the haze extinction profiles from the intensity observations, we model the synthetic
radiance factor ($I/F$) with a single scattering ray tracing model in a spherical shell geometry.
The effect of multiple scattering is accounted for as a correction $\varrho_k (z)$
applied to the volume scattering along line of sight.
This technique was used successfully several times before \citep[\eg][]{Rages1983, Rannou1997, Seignovert2017, West2018}.

In the detached haze layer, the multiple scattering is mainly produced by the light coming from the main haze below. To
evaluate $\varrho_k$, we use a representative vertical profile of the main haze that
reproduce the observed intensity of Titan in the UV. With a radiative transfer model \citep[SHDOMPP, from ][]{Evans1998}
we have access to the complete radiative source function at each level of the atmosphere as well as the optical properties (haze absorption and scattering and Rayleigh scattering). We are then able to compare the
intensity that is scattered in the direction of the observer from the direct sun only and from the direct sun and the scattered
field coming from below. $\varrho_k$ is defined as the ratio of multiple scattering to single scattering toward the observer
for a given altitude and as a function of the incident and emergent angles. This parameter is pre-computed as a function of
altitude, incident and emergent angles and saved in a look-up table \citep[see.][for details]{West2018}.
We find that the multiple scattering increases the scattered intensity at the limb of Titan in the UV by a ratio between
1.05 and 1.20, depending on the geometry of the observation.
This effect is included in our model. We also checked that significant changes in the main haze (single scattering albedo, the opacity and the vertical scale height) only affect the value of $\varrho_k$ by a few percent and can be neglected.

We discretize the atmosphere in $N = 60$ irregular layers of various thickness : $\Delta z =$ 50 km from the
ground to 200 km, $\Delta z =$ 25 km from 200 to 300 km, $\Delta z =$ 10 km from 300 to 400 km and from 550 to 700 km.
Finally, we used $\Delta z$= 5 km between 400 and 550 km. This grid allows us to take advantage of the spatial resolution
of the ISS NAC camera in the region of interest where is mainly located the DHL.

We can write the outgoing $I/F (z)$ as:

\begin{equation}
I/F (z) = \sum_{i=0}^{n_x-1} \int\limits_{x_k}^{x_{k+1}}
\frac{\left< \varpi P(\Theta) \right>_k}{4}
e^{-\left( \tau^i_k\left(z\right) + \tau^e_k\left(z\right) \right)}
\beta_k\left(z\right) \varrho_k\left(z\right) d{x}
\label{eq:west2017_sup_limb}
\end{equation}

where the summation is performed on the $n_x-1$ segments defined by the intersections of the line of sight and the spherical shell boundaries. The impact parameter $z$ (the lowest altitude reached by the line of sight) is given by the bottom of the $n^\mathrm{th}$ layer crossed. Therefore, each layer of the atmosphere is crossed twice. $x$ is the abscissa along the line of sight. $\tau^i_k$ and $\tau^e_k$ are the opacities along the incident and emergent paths.
$\left< \varpi P(\Theta)\right>_k$ is the average of the product of the single scattering and the phase function at the scattering angle of the observation $\Theta$ for the layer crossed on $x_k$.  $\beta_k(z)$ is the local extinction coefficient at altitude $z$ (and the product of the haze cross-section and the local number density). Here, the altitude $z(x)$ is the local altitude at point of abscissa $x$ along the line of sight.


\begin{figure*}[!ht]
\includegraphics[width=.4\textwidth]{Fig/N1551888681_sampling}
\includegraphics[width=.57\textwidth]{Fig/Model_uncertainties}
\caption{Left: Log scale representation of the $I/F$  for the image
\textbf{N1551888681\_1} taken in March, 2007. The data are sampled at the limb by
bins of \ang{5}. The red area correspond to the bin at the photometric
equator ([\ang{45}S, \ang{50}S] and [\ang{76}W, \ang{79}W]). The pixel scale
(7.9 km) is represented in the zoomed area by the yellow rectangle.
An altitude grid is also represented where we can see the peak of intensity of the DHL
located at 500 km. Right: the red curve represents the inverted $I/F$ profile fitted
on the resampled data (orange error bars) within a \ang{5} bin in latitude.
The uncertainties on the retrieved extinction profile $\beta$ (in blue) is calculated to match the $1 \sigma$ distribution of the $I/F$ pixels. The model fits the observations down to 300 km.
The uncertainties increase rapidly when observed ($I/F$) is close
to the noise level ($3\cdot10^{-3}$).}
\label{fig:model_uncertainties}
\end{figure*}


\subsection{Retrieval method}

Based on our previous work \citep{Seignovert2017, West2018}, we make the assumption that the optical properties $\left<\varpi P(\Theta)\right>_k$ of the aerosols are constant in the upper part of Titan's atmosphere. This allows us to focus our study only on the retrieval of the extinction along the line of sight.
In our model, the $I/F (z)$ intensity profiles depend on the set haze extinction profile $\beta(z)$ and the viewing geometry of the observation (incidence, emergence and phase angles).
We assume no horizontal in-homogeneity along the line of sight \citep{Seignovert2017}. We retrieve a set of extinction values $\beta_i$ (the vector ${\beta}$), with $i$ the indices of the layers, that matches the values of the $I/F_i$ (the vector $I/F$).

From the Eq. (\ref{eq:west2017_sup_limb}), it is possible to cast the scattered intensity $I/F_i$, as a function of the extinction $\beta_j$ with $j \le i$. This forms a non-linear triangular system. To find the vector $\beta$, we have to solve the formal equation:

\begin{equation}
    I/F = G(\beta)
\end{equation}

where ${G}$ is a nonlinear function which depends on $\beta_i$ and on the viewing geometry of the observation.

We solve the system by minimizing globally the difference between the modeled $I/F$ and the observations
using a Levenberg-Marquardt minimization. Therefore, we obtain simultaneously all the $\beta_i$ at once.
Moreover, the tangential opacity along a line of sight ($\tau_{los}$) is considered opaque when it reaches 3 (usually around 300 km in the UV).
Beyond this threshold, we do not retrieve the value of $\beta$.
An example of inversion is presented in the figure~\ref{fig:model_uncertainties}.

The uncertainties associated to the intensity profile ($I/F$) correspond to a $1 \sigma$ distribution of the observed $I/F$ for a given altitude within a \ang{5} bin in latitude.
The number of pixels within a \ang{5} bin varies from one image to another and depends on the viewing geometry and the altitude sampled. Usually we have more than 1500 pixels per profile between 0 and 700 km.
In the worse case scenarios (N1630432142\_1 at $6 \sigma$ of the average pixel scale) we have at least 120 pixels per profile.
This allow us to resampled our observed profile on a vertical grid with layers smaller than the image pixel scale in the 500 km region.
The uncertainties in the extinction profile ($\beta$) correspond to the minimum and maximum values required to fit the data within $1 \sigma$ of the observed $I/F$ in each layer independently \citep[contrary to][ where the errors were computed in all the layers simultaneously]{West2018}.
