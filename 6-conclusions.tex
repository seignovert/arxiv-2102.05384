\section{Conclusions}

In this article we have analyzed observations of light scattered at the limb of Titan with ISS (UV3 filter) onboard
Cassini. We retrieved the haze extinction as a function of altitude, latitude and time between October 2004 and
September 2017. We followed, during about half Titan's year, the evolution of the DHL and the top of the
main haze. Especially, we witnessed the collapse of the DHL during the equinoctial transition of the atmosphere
circulation and its reappearance before the following solstice.

We confirmed and gave details about the collapse of the detached haze layer previously reported by \cite{West2011}.
We also tracked the small scale variations after the equinox in order to detect the reappearance of the detached haze
layer at the end of 2015 \citep{West2018}. These two previous works focused on the DHL at equator. Here, we could give
a full description of the structures of the haze layer as a function of altitude, latitude and time. We can remark
that the DHL has a natural variability, which can be temporal and spatial, and sometimes two distinct hazes or plumes
can be observed above the DHL. The amount of data is not large enough to really distinguish between spatial or
longitudinal variability. But data taken with a polar viewing clearly indicates that the haze is not completely
uniform in longitude.

The equinoctial collapse clearly starts in the summer hemisphere. Its initial phase can be discerned in March
2008. The main haze collapses first and then the detached haze layer about one terrestrial year later. In April
2012, the detached haze is below 300 km and can not be seen at UV wavelengths.
The fall of the detached haze layer between 2009 and 2011 occurred at the average speed of -67 km/yr. At two moments,
during the equinoctial collapse, the DHL seems to settle down at the aerosol terminal speed
\citep[as reported by][]{West2018}. If this is real, it would indicate an absence of vertical wind at these two moments.
During a period of 3 years and a half (from mid-2011 to 2015), no stable detached haze layer showed up.
However, the haze layer fluctuates, and sporadic local detached layers appeared and disappeared rapidly.

The detached haze layer reappear in observations of December 2015. This new detached haze is much fainter than expected.
The timing of the reappearance is quite similar to the predictions of GCMs but it occurs with much complex patterns
that those predicted. First, it reappeared around 500 km in December 2015 ($L_s =\ang{75}$) as a very faint
structure which became persistent and more pronounced with time. At the equator this structure sedimented and finally disappeared in about one year whereas it remains visible in the northern hemisphere.
We found that the equatorial fall speed of the detached haze layer in 2009 and in 2016 are similar, around -67 km/yr.
A second detached haze appeared in July 2016 around 500 km, above the first DHL, and apparently started to settle down
as well. Unfortunately, the survey was interrupted in September 2017 by the end of Cassini mission. This second detached
haze layer did not cover all the globe and was quite variable, although it was present up to the final observation.

Unfortunately, no data were acquired between April 2008 and February 2009 and between December 2010 and September 2011.
We thought to use the NAC UV1 and UV2 images but these data have a very poor signal to noise ratio and
can not be included in our analysis.
However, a few images are available with the WAC camera in the VIO (Violet), BL1 (Blue) filters.
The behavior of the aerosols at these wavelength should be very similar to the UV3 filter and could fill these gaps.
It would also be interesting to perform similar analysis with data acquired through filters at larger wavelengths.
In these case, it would be possible to probe deeper layers in order to monitor the collapse of the DHL further
down and, as well, the cycle of the main haze. For instance, \cite{Rages1983} could probe as lower as 200-250 km
in clear filter ($\lambda_{eff} \simeq 0.5 \mu$m). At even longer wavelengths, we could reach levels in the low
stratosphere and, maybe probe high altitude polar clouds.

Comparison with General Circulation Models are fruitful. Our results reinforce, the scenario of a
detached haze cycle primarily controlled by circulation as proposed by \cite{Toon1992} with a 1D model, \cite{Rannou2002}
with a coupled 2D-GCM and \cite{Lebonnois2012, Larson2015} with coupled 3D-GCM. Although GCMs capture the global
haze cycle, many differences remain, mainly driven by technical limitations.